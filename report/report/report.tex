{\centering{\chapter*{РЕФЕРАТ}}}
Отчет 44 с., 1 кн., 8 рис., 20 источн. \newline
КЛАСТЕРИЗАЦИЯ РАДИОСИГНАЛОВ, МЕТОДЫ КЛАСТЕРИЗАЦИИ, МОДЕЛЬ ГАУССОВОЙ СМЕСИ, МАКСИМИЗАЦИЯ ОЖИДАНИЯ, МОДИФИЦИРОВАННЫЙ АЛГОРИТМ КЛАСТЕРИЗАЦИИ, ЛОКАЛЬНАЯ ПЛОТНОСТЬ

Объект исследования --- задача кластеризации. Предмет исследования --- кластеризация радиосигналов РЛС, отраженных от объектов.

Цель работы --- разработать метод кластеризации отраженных радиолокационных сигналов.

В работе формулируется задача кластеризации отраженных радиолокационных сигналов. Проводится обзор существующих методов кластеризации. Формулируются критерии сравнения методов и выполняется сравнительных анализ. Приводится функциональная модель разработанного метода, а также описание каждого из пяти этапов работы. Ключевые шаги метода сопровождаются схемами алгоритма. Приведена реализация разработанного метода. Проведено исследование эффективности разработанного метода кластеризации.

Результатом работы является разработанный метод кластеризации отраженных радиолокационных сигналов.

Область применения результатов --- кластеризация данных об отраженных радиолокационных сигналах.