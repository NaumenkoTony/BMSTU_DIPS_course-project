\section{Сценарии функционирования системы}

\subsection{Авторизация пользователя}
\begin{enumerate}
	\item Пользователь нажимает на кнопку <<Войти в систему>> на фронтенде.
	\item Пользователь перенаправляется на сервис авторизации, предоставляющий страницу ввода логина и пароля.
	\item Пользователь вводит логин и пароль, затем подтверждает ввод нажатием кнопки <<Войти>>.
	\item Если данные указаны неверно, отображается сообщение об ошибке.  
	При корректных данных пользователь получает доступ к системе и попадает на главную страницу.
\end{enumerate}

\subsection{Просмотр списка отелей и бронирование}
\begin{enumerate}
	\item На главной странице отображается список отелей, разбитый на страницы.  
	Пользователь может выбрать номер страницы и количество записей на ней.  
	Таблица включает 6 столбцов: страна, город, название отеля, количество звезд, цена, а также действие (кнопка <<Забронировать>>).
	\item Доступна сортировка и фильтрация списка в рамках страницы.  
	\begin{itemize}
		\item Сортировка: по стране, городу, названию отеля (по алфавиту), количеству звезд, цене (по возрастанию/убыванию).  
		\item Фильтрация: по стране, городу и названию отеля (совпадение подстроки), по количеству звезд (не менее заданного), по цене (не более заданной суммы).
	\end{itemize}
	\item Для бронирования пользователь нажимает кнопку <<Забронировать>> в строке нужного отеля.  
	В открывшейся форме он указывает даты заезда и выезда и подтверждает действие кнопкой <<Забронировать>>.  
	При успешном бронировании отображается сообщение с итоговой стоимостью, а также кнопки <<Отменить>> и <<Мои бронирования>>.
\end{enumerate}

\subsection{Просмотр списка бронирований}
\begin{enumerate}
	\item После авторизации пользователь нажимает кнопку <<Бронирования>> в верхней части страницы.
	\item Открывается страница со списком бронирований в виде карточек.  
	На карточке отображаются: название отеля, даты заезда и выезда, итоговая стоимость и статус оплаты.
	\item Карточки активных бронирований (дата заезда еще не наступила) содержат кнопку <<Отменить бронирование>>.  
	В зависимости от текущей даты карточки имеют разные статусы и визуальное оформление: <<Отменено>>, <<Запланировано>>, <<Скоро>>, <<Сейчас в отеле>>, <<Завершено>>.
\end{enumerate}

\subsection{Просмотр информации о профиле}
\begin{enumerate}
	\item После авторизации пользователь нажимает кнопку <<Профиль>>.
	\item Открывается страница профиля, где отображаются: логин, адрес электронной почты, роль в системе, имя и фамилия.
	\item Дополнительно выводится информация о программе лояльности: размер скидки, количество совершенных бронирований и необходимое количество для повышения статуса.
\end{enumerate}
