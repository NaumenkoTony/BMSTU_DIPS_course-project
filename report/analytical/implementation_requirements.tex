\section{Требования к реализации}
К реализации системы предъявляются следующие требования.
\begin{enumerate}
	\item В разрабатываемой системе пользователи делятся на две роли: Пользователь и Администратор.
	\item Требуется использовать сервис-ориентированную архитектуру.
	\item Все бекенды и фронтенд должны быть запущены изолированно друг от друга.
	\item Для межсервисного взаимодействия использовать HTTP (RESTful API).
	\item Для запросов, выполняющих обновление данных на нескольких узлах распределенной системы, в случае недоступности одной из систем, необходимо выполнять полный откат транзакции.
	\item Необходимо реализовать пользовательский интерфейс как Single Page Application для фронтенда. Интерфейс должен быть доступен через тонкий клиент --- браузер.
	\item Серверы бекендов недоступны пользователю, это реализуется их расположением во внутренней сети.
	\item Доступ к разрабатываемым базам данных должен осуществляться по протоколу TCP.
	\item Валидация входных данных должна производиться на стороне фронтенда. Бекенды не должны валидировать входные данные, так как пользователь не может к ним обращаться напрямую, то есть бекенды должны получать уже отфильтрованные входные данные от фронтенда.
	\item Разрабатываемая система должна поддерживать возможность горизонтального и вертикального масштабирования за счет увеличения количества функционирующих узлов и совершенствования технологий реализации компонентов и всей ее архитектуры.
\end{enumerate}