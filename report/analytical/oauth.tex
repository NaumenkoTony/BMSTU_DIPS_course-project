\section{OAuth Authorization Flow}

\subsection{Сценарий авторизации}
Authorization code Flow определен в OAuth 2.0 RFC 6749, раздел 4.1. Он включает в себя обмен кода авторизации на \mbox{токен \cite{oauth}.} 
\begin{enumerate}
	\item Пользователь выбирает опцию <<Войти>> в приложении.
	\item Пользователь перенаправляется на сервер авторизации Auth0 (endpoint /authorize).
	\item Сервер авторизации Auth0 отображает страницу аутентификации и при необходимости окно согласия с перечнем разрешений, предоставляемых приложению.
	\item Пользователь проходит аутентификацию одним из доступных способов.
	\item Сервер авторизации Auth0 перенаправляет пользователя обратно в приложение, передавая одноразовый код авторизации.
	\item Auth0 отправляет код авторизации, идентификатор клиента и учетные данные приложения (например, client secret или Private Key JWT) на сервер авторизации Auth0 (endpoint /token).
	\item Сервер авторизации Auth0 проверяет код авторизации, идентификатор клиента и его учетные данные.
	\item В случае успеха сервер авторизации возвращает ID Token, Access Token, при необходимости --- Refresh Token.
	\item Приложение использует Access Token для вызова API и получения информации о пользователе.
	\item API проверяет токен и отвечает приложению запрошенными данными.
\end{enumerate}

\subsection{PKCE для публичных клиентов}
Для клиентов, работающих на стороне фронтенда (например, SPA или мобильных приложений), использование client secret невозможно, так как его нельзя хранить безопасно.  
В таких случаях применяется расширение PKCE (Proof Key for Code \mbox{Exchange) \cite{pkce}.}

При инициализации запроса авторизации приложение формирует случайную строку --- code verifier, из которой вычисляется code challenge (с помощью SHA-256 и Base64URL).  
\begin{itemize}
	\item В параметрах запроса к endpoint /authorize передается code challenge.  
	\item После получения кода авторизации клиент направляет запрос к endpoint /token, указывая исходный code verifier.  
	\item Сервер авторизации сверяет соответствие code challenge и code verifier.  
\end{itemize}
Таким образом, снижается риск подмены кода авторизации злоумышленником.

\subsection{Проверка токенов с использованием JWKS}
Возвращаемые сервером токены (ID Token и Access Token) подписаны приватным ключом.  
Для проверки их подлинности приложение и API используют открытые ключи, публикуемые сервером авторизации в формате JWKS (JSON Web Key \mbox{Set) \cite{jwks}.}  

Проверка токена включает:  
\begin{itemize}
	\item извлечение заголовка токена и определение идентификатора ключа (kid);  
	\item получение соответствующего открытого ключа из JWKS-документа по URL сервера авторизации;  
	\item проверку криптографической подписи токена;  
	\item проверку срока действия (\texttt{exp}) и других обязательных полей (например, \texttt{aud}, \texttt{iss}).  
\end{itemize}

Данный механизм гарантирует, что токен действительно выдан сервером авторизации и не был изменен.
