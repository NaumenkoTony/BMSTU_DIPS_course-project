{\centering{\chapter*{ВВЕДЕНИЕ}}}
\addcontentsline{toc}{chapter}{ВВЕДЕНИЕ}
Системы бронирования отелей устраняют ключевую проблему туристического рынка --- неэффективный поиск и оформление размещения. Они
автоматизируют процесс бронирования, предоставляя туристам актуальную информацию о номерах и ценах, а отелям --- удобный инструмент управления
процессом бронирования. Такие системы упрощают взаимодействие туристов и гостиниц, способствуя развитию туристической отрасли в целом.

Распределенная система --- это набор компьютерных программ, использующих вычислительные ресурсы нескольких отдельных вычислительных узлов для достижения одной общей цели. Распределенная система основывается на отдельных узлах, которые обмениваются данными и выполняют синхронизацию в общей сети. Распределенные системы направлены на устранение узких мест или единых точек отказа в системе. Для системы бронирования отелей данный подход является наиболее подходящим, поскольку позволяет обеспечить высокую доступность и надежность. Микросервисная архитектура обеспечивает масштабируемость и повышает безопасность системы в целом.

\textbf{Цель} работы --- Разработать прототип системы поиска и бронирования отелей на интересующие даты.

\textbf{Необходимо выполнить следующие задачи.}
\begin{enumerate}[topsep=0pt]
	\item Представить обзор существующих сервисов бронирования отелей, описать требования к разрабатываемой распределенной системе и сформулировать ее бизнес-логику.
	\item Спроектировать и описать архитектуру системы, а также выделенные в ней сущности. Представить схему взаимодействия сервисов с помощью диаграммы последовательности действий.
	\item Разработать программное обеспечение, описать типы, структуры данных, а также процесс сборки и развертывания системы..
\end{enumerate}