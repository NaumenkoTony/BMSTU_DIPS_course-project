\section{Тестирование системы}
Тестирование системы интегрировано в GitHub Actions и выполняется с использованием Postman и утилиты newman, позволяющей автоматизировать прогон API-тестов. 

На первом этапе, после сборки контейнеров и запуска окружения, тесты проверяют как базовую доступность API, так и корректность деградации функциональности при недоступности отдельных сервисов. Таким образом, обеспечивается контроль не только стандартных сценариев работы, но и поведения системы в условиях отказа некритичных компонентов.

На втором этапе, после развертывания микросервисов в кластере Kubernetes, тесты выполняются повторно. Здесь проверяются типичные сценарии использования API в условиях, максимально приближенных к production-среде. Такой подход позволяет убедиться, что система функционирует корректно после развертывания и что ее базовые возможности остаются доступными пользователям.
