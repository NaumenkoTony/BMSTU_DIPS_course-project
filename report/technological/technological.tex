\chapter{Технологический раздел}

\section{Средства реализации приложения}
Программное обеспечение реализовано на языке Python версии 3.12.10, который на сегодняшний день является основным и наиболее удобным языком программирования для задач машинного обучения и анализа данных. Благодаря наличию специализированных библиотек Python стал универсальным выбором для решения задач в области машинного обучения и искусственного интеллекта.

В рамках реализации использовались библиотеки:
\begin{itemize}
	\item \texttt{NumPy} --- библиотека для численных расчетов;
	\item \texttt{scikit-learn} --- реализация алгоритмов кластеризации, метрик качества и масштабирования признаков;
	\item \texttt{SciPy} --- статистические функции и меры сходства.
\end{itemize}

Разработка велась с использованием Jupyter Notebook и Visual Studio Code.

\section{Реализация этапа вычисления локальной плотности и фильтрации шумов}
В листинге \ref{lst:compute_local_densities_with_selection.py} представлена реализация этапа вычисления локальной плотности и фильтрации шумов.

\includelisting
{compute_local_densities_with_selection.py}
{Реализация этапа вычисления локальной плотности и фильтрации шумов.}

\section{Реализация этапа определения количества кластеров}
В листинге \ref{lst:find_optimal_clusters.py} представлена реализация этапа определения количества кластеров.

\includelisting
{find_optimal_clusters.py}
{Реализация этапа определения количества кластеров.}

\section{Реализация модифицированного метода кластеризации}
В листинге \ref{lst:modified_em_clustering.py} представлена реализация модифицированного метода кластеризации.

\includelisting
{modified_em_clustering.py}
{Реализация модифицированного метода кластеризации.}

\section{Реализация разрешения взаимных наложений кластеров}
В листинге \ref{lst:resolve_cluster_overlap.py} представлена реализация этапа разрешения взаимных наложений кластеров.

\includelisting
{resolve_cluster_overlap.py}
{Реализация этапа разрешения взаимных наложений кластеров.}