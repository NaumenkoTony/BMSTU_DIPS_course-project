\section{Описание функциональных сервисов системы}
Система включает в себя фронтенд, сервис-координатор и 5 подсистем:
\begin{enumerate}
	\item сервис аутентификации;
	\item сервис лояльности;
	\item сервис платежей;
	\item сервис бронирований;
	\item сервис сбора статистики.
\end{enumerate}

Все сервисы должны принимать и отдавать данные в формате JSON по протоколу HTTP.

\subsection{Фронтенд}
Фронтенд --- серверное приложение, предоставляет пользовательский интерфейс и внешний API системы. При его разработке нужно учитывать следующие факторы:
\begin{itemize}
	\item фронтенд должен принимать запросы по протоколу HTTP и формировать ответы пользователям в формате HTML;
	\item фронтенд должен отправлять поступающие от пользователей запросы в сервис-координатор;
	\item запросы к бекендам необходимо осуществлять по протоколу HTTP;
	\item данные необходимо передавать в формате JSON;
	\item фронтенд должен предоставлять пользовательский интерфейс как Single Page Application.
\end{itemize}

\subsection{Сервис координатор}
При его разработке нужно учитывать, что сервис координатор должен:
\begin{itemize}
	\item координировать обмен сообщениями между сервисами внутри системы, направляя запросы к соответствующим бекенд серверам и отправляя их ответы на фронтенд;
	\item накапливать статистику запросов, и в случае, если какой-то из бекендов не ответил N раз, то на N + 1 раз вместо запроса к нему отдавать fallback-ответ, а через некоторое время выполнять запрос к этому бекенду, чтобы проверить его состояние (реализация паттерна Circuit Breaker).	
\end{itemize}

\subsection{Сервис аутентификации}
При его разработке нужно учитывать, что сервис аутентификации должен:
\begin{itemize}
	\item выполнять функцию Identity Provider;
	\item реализовывать протокол OpenId Connect Authorization Flow.
\end{itemize}
СУБД сервиса аутентификации должна хранить следующие сущности.
\begin{enumerate}
	\item Код авторизации.
		\begin{enumerate}
		\item Идентификатор кода (guid, генерируется автоматически, первичный ключ).
		\item Идентификатор клиента (строка, обязательное поле, ссылка на сущность \textit{Клиент}).
		\item Идентификатор пользователя (строка, обязательное поле).
		\item Конечная точка перенаправления (строка, не более 256 символов, обязательное поле).
		\item Список запрошенных прав доступа (строка, обязательное поле).
		\item Срок действия кода (дата и время, обязательное поле).
		\item Кодовое значение для PKCE (строка).
		\item Метод преобразования для PKCE (строка).
	\end{enumerate}
	\item Клиент.
		\begin{enumerate}
		\item Идентификатор записи (строка, генерируется автоматически, первичный ключ).
		\item Идентификатор клиента (строка, обязательное поле, уникальное значение).
		\item Получатель (строка, обязательное поле).
		\item Секрет клиента (строка, указывается только для конфиденциальных клиентов).
		\item Список разрешенных redirect URI (строка, обязательное поле).
		\item Разрешенные права доступа (строка, обязательное поле).
		\item Требуется использование PKCE (логическое значение).
		\item Признак публичного клиента (логическое значение).
	\end{enumerate}
\end{enumerate}

\subsection{Сервис лояльности}
При его разработке нужно учитывать, что он должен:
\begin{itemize}
	\item выполнять определение величины скидки по идентификатору пользователя;
	\item выполнять обновление числа забронированных отелей и статуса в программе лояльности (предусмотреть, как повышение, так и понижение в случае отмены бронирования).
	\item отправлять события о действиях в брокер сообщений kafka.
\end{itemize}
СУБД сервиса лояльности должна хранить следующие сущности.
\begin{enumerate}
	\item Лояльность.
	\begin{enumerate}
		\item Идентификатор (целое число, генерируется автоматически, первичный ключ).
		\item Логин пользователя (строка, обязательное поле).
		\item Количество бронирований (целое число, значение по умолчанию – 0).
		\item Статус (строка, бронзовый/серебряный/золотой, по умолчанию – бронзовый).
		\item Размер скидки (целое число).
	\end{enumerate}
\end{enumerate}

\subsection{Сервис платежей}
При его разработке нужно учитывать, что он должен:
\begin{itemize}
	\item предоставлять информацию о платежной операции по ее идентификатору;
	\item создавать запись об оплате;
	\item выполнять получение и обновление статуса оплаты.
	\item отправлять события о действиях в брокер сообщений kafka.
\end{itemize}
СУБД сервиса платежей должна хранить следующие сущности.
\begin{enumerate}
	\item Платеж.
	\begin{enumerate}
		\item Идентификатор платежа (целое число, генерируется автоматически, первичный ключ).
		\item Идентификатор записи (guid).
		\item Сумма (целое число).
		\item Статус платежа (строка).
	\end{enumerate}
\end{enumerate}

\subsection{Сервис бронирований}
При его разработке нужно учитывать, что он должен:
\begin{itemize}
	\item предоставлять информацию об отелях;
	\item создавать запись о бронировании;
	\item предоставлять информацию о бронированиях конкретного пользователя;
	\item выполнять получение и обновление статуса бронирования.
	\item отправлять события о действиях в брокер сообщений kafka.
\end{itemize}
СУБД сервиса бронирований должна хранить следующие сущности.
\begin{enumerate}
	\item Отель.
	\begin{enumerate}
		\item Идентификатор (целое число, генерируется автоматически, первичный ключ).
		\item Глобальный уникальный идентификатор отеля (guid, генерируется автоматически, уникальное поле).
		\item Название (строка, не более 255 символов, обязательное поле).
		\item Страна (строка, не более 80 символов, обязательное поле).
		\item Город (строка, не более 80 символов, обязательное поле).
		\item Адрес (строка, не более 255 символов, обязательное поле).
		\item Количество звезд (целое число от 1 до 5, может быть пустым).
		\item Цена (целое число, обязательное поле).
	\end{enumerate}
	\item Бронирование.
	\begin{enumerate}
		\item Идентификатор (целое число, генерируется автоматически, первичный ключ).
		\item Глобальный уникальный идентификатор бронирования (guid, генерируется автоматически, уникальное поле).
		\item Имя пользователя (строка, не более 80 символов, обязательное поле).
		\item Идентификатор платежа (guid, обязательное поле).
		\item Идентификатор отеля (целое число, внешний ключ на сущность Отель).
		\item Статус (строка, не более 20 символов, обязательное поле, возможные значения: <<Не оплачено>>, <<Оплачено>>, <<Отменено>>).
		\item Дата заселения (дата, обязательное поле).
		\item Дата выезда (дата, обязательное поле).
	\end{enumerate}
\end{enumerate}

\subsection{Сервис сбора статистики}
При его разработке нужно учитывать, что он должен:
\begin{itemize}
	\item принимать события о действиях в системе с помощью брокера сообщений kafka;
	\item сохранять и предоставлять информацию о действиях в системе.
\end{itemize}
СУБД сервиса сбора статистики должна хранить следующие сущности.
\begin{enumerate}
	\item Действие пользователя.
	\begin{enumerate}
		\item Идентификатор (guid, генерируется автоматически, первичный ключ).
		\item Идентификатор пользователя (строка, не более 128 символов, обязательное поле).
		\item Имя пользователя (строка, не более 128 символов, обязательное поле).
		\item Название сервиса (строка, не более 128 символов, обязательное поле).
		\item Действие (строка, не более 128 символов, обязательное поле).
		\item Статус (строка, не более 128 символов, обязательное поле).
		\item Метка времени (дата и время с часовым поясом, обязательное поле).
		\item Дополнительные данные (json).
		\item Топик (строка, не более 128 символов, обязательное поле).
		\item Раздел (целое число, обязательное поле).
		\item Смещение (целое число, обязательное поле, уникальность обеспечивается в связке с топиком и разделом).
	\end{enumerate}
\end{enumerate}



